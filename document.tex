%Kleiner Text, zweispaltig
\documentclass[twocolumn, 10pt]{article}

\usepackage[ngerman]{babel}
\usepackage[utf8]{inputenc}

%Mathe symbole
\usepackage{amsmath}
\usepackage{amssymb}

%Ränder Anpassen 
\usepackage{scrextend}
\usepackage[hmargin=.3cm,vmargin=1.2cm,showframe=false]{geometry}

%Abstand zwischen Listeneinträgen verringern
\usepackage{enumitem}
\setlist{parsep=-.3em}
%\setlist{itemsep=-.5em}

 
 
%Definitionsumgebung
\newenvironment {definition}
                [1][]
                {\noindent\\{\bf
                Def:}\emph{
                #1}\indent\begin{addmargin}{.5em}}{\end{addmargin}}

%Schönere Befehle für logische operatoren
\newcommand{\und}{\wedge}
\newcommand{\oder}{\vee}

\begin{document}

\title{Algebra Cheat Sheet}
\maketitle



\subsection*{Permutationen}

{\tiny Seite 41}
\begin{definition}[Permutation]
Ist eine bijektive Abbildung $\alpha :X\rightarrow X$.\\
$S_X := $ Menge aller Permutationen von X\\
$S_n := $ $S_X$ mit $X=\{1, 2,\ldots ,n\}$
\end{definition}

\begin{definition}[r-Zykel]
Seien $i_1,\ldots ,i_r \in \{1, 2,\ldots ,n\}$ paarweise verschieden.\\
Ein Zykel der Länge r ist dann $\alpha :=(i_1i_2\ldots i_r) \in S_n$ \\mit
$\alpha (i_1)=i_2$, $\alpha (i_2)=i_3$,\ldots $\alpha (i_r)=i_1$.
$(a_0a_1a_2a_3a_4) \rightarrow (a_4a_1a_2a_3a_0)$
\end{definition}

\begin{definition}[Transposition]
Ein 2-Zykel
\end{definition}



\subsection*{Morphismen}

{\tiny Seite 50}
\begin{definition}[Homomorphismus/Morphismus]
Gegeben sind zwei Halbrguppen: $(G, \times )$, $(H, \# )$. Die Abbildung
$f:G \rightarrow H$ ist ein Homomorphismus, wenn
$f(a\timesb )=f(a)\# f(b)\hspace{.5cm}\forall a,b \in G$.
\end{definition}

\begin{definition}[Monomorphismus]
Ein injektiver Homomorphismus
\end{definition}

\begin{definition}[Epimorphismus]
Ein surjektiver Homomorphismus
\end{definition}

\begin{definition}[Isomorphismus]
Ein bijektiver Homomorphismus
\end{definition}

\begin{definition}[Automorphismus]
$f:G \rightarrow G$, $f$ injektiv: beide Halbgruppen sind gleich.
\end{definition}



\subsection*{Misc}
\begin{definition}[injektiv]
$f(a)=f(a') \Rightarrow a=a'$ \glqq für jedes x max. ein y\grqq\\
$f:A\rightarrow B$ ist \emph{injektiv} genau dann wenn $\forall b \in B:
|f^-1(\{b\})| \leq 1$
\end{definition}

\begin{definition}[surjektiv]
$\forall b \in B, \exists a \in A: b = f(a)$ \glqq die gesamte Bildmenge ist
erreichbar\grqq
\end{definition}

\begin{definition}[bijektiv]
Injektiv und Surjektiv zusammen.
\end{definition}

\begin{definition}[Bild]
$f^-1(C):{a \in A: f(a) \in C}$
\end{definition}



\subsection*{Gruppen}

\begin{definition}[Halbgruppe (s. 37)]
\begin{itemize}
  \item Trägermenge G
  \item Abbildung: $G \times G \rightarrow G$, assoziativ
\end{itemize}
\end{definition}

\begin{definition}[Monoid (s. 37)]
\begin{itemize}
  \item Trägermenge G
  \item Abbildung: $G \times G \rightarrow G$, assoziativ, kann kommutativ
  (abelsch) sein
  \item Neutrales Element $a^0 = e$
\end{itemize}
\end{definition}

\begin{definition}[Gruppe (s. 47)]
\begin{itemize}
  \item Trägermenge G
  \item Abbildung: $G \times G \rightarrow G$, assoziativ, kann kommutativ
  (abelsch) sein
  \item Neutrales Element $a^0 = e$
  \item Inverses Element $a^-1$
\end{itemize}
\end{definition}

\begin{definition}[Ring (s. 85)]
\begin{itemize}
  \item Trägermenge G
  \item Abbildungen
  	\begin{enumerate}
    	\item $"+": G\times G \rightarrow G$, assoziativ, distributiv, kommutativ,
    	neutrales Element \glqq 0\grqq , inverses Element \glqq -a\grqq
    	\item $"\cdot ": G\times G \rightarrow G$, assoziativ, distributiv,
    	optional kommutativ, optinal neutrales Element $\neq 0$ (\glqq
    	Einselement\grqq ), optional inverses Element (\glqq Einheit\grqq )
  	\end{enumerate}
\end{itemize}
\end{definition}

\begin{definition}[Körper (Wikipedia)]
\begin{itemize}
  \item Trägermenge G
  \item Abbildungen
  	\begin{enumerate}
    	\item $"+": G\times G \rightarrow G$, assoziativ, distributiv, kommutativ,
    	neutrales Element \glqq 0\grqq, inverses Element \glqq -a\grqq
    	\item $"\cdot ": G\times G \rightarrow G$, assoziativ, distributiv,
    	kommutativ, neutrales Element $\neq 0$ (\glqq Einselement\grqq ), inverses
    	Element (\glqq Einheit\grqq )
  	\end{enumerate}
\end{itemize}
\end{definition}



\subsection*{Verbände und Boolsche Algebren}

\begin{definition}[Teilweise geordnete Menge (s. 10)]
Eine Menge ist teilweise Geordnet, wenn auf ihr eine Relation $\leq$ existiert,
die \begin{itemize}
  \item reflexiv ($a\leq a$)
  \item antisymmetrisch ($a \leq b \text{ und } b\leq a \Rightarrow b = a$)
  \item transitiv ($a\leq b \text{ und } b\leq c \Rightarrow a\leq c$)
\end{itemize}
ist. Die Relation $<$ bedeutet dann $\leq$ und $a\neq b$.
\end{definition}

\begin{definition}[Total geordnete Menge/Kette (s. 10)]
Eine Menge ist total geordnet, wenn sie Teilweise geordnet ist und für alle
Elemente x, y der Menge gillt entweder $x \leq y$ oder $y \leq x$.
\end{definition}

\begin{definition}[Speziele Elemente in teilweise geordneten Mengen (s. 10)]
Sei $(P, \leq)$ teilweise geordnet, $X \subset P$, $X \neq \emptyset$
\begin{enumerate}
  \item $y \in X$ ist \emph{minimal}, wenn $\forall x \in X: x > y$
  \item $y \in X$ ist \emph{kleinstes Element}, wenn $\forall x \in X: y \leq x$ 
  \item $y \in P$ ist \emph{untere Schranke} von $X$, wenn $\forall x \in X: y
  \leq x$
  \item $y \in P$ ist \emph{größte Untere Schranke} von $X$, wenn $y$ größtes
  Element in der Menge der unteren schranken ist.
\end{enumerate}
Bei total geordneten mengen gillt:\\\indent \emph{minimales Element} =
\emph{kleinstes Element}
\end{definition}

\begin{definition}[Verband (s. 16)]
Ein Verband ist eine teilweise geodnete Menge. je zwei Elemente x, y der Menge
haben:
\begin{itemize}
  \item Durchschnitt $:= x \und y :=Min\{ x, y\} :=$ größte untere Schranke von
  \{x, y\} , falls sie existiert.
  \item Vereinigung $:=x \oder y :=Max\{ x, y\} :=$ kleinste obere Schranke von
  \{x, y\} , falls sie existiert.
\end{itemize}
\emph{Rechenregeln:}
\begin{itemize}
  \item $x \und x = x$, $x \oder x = x$
  \item $(x \und y) \und z = x \und (y \und z)$, $(x \oder y) \oder z = x \oder (y
  \oder z)$
  \item $x \und y = y \und x$, $x \oder y = y \oder x$
  \item $x \und (x \oder y) = x$, $x \oder (x \und y) = x$
  \item $x \leq y \Leftrightarrow x \und y = x \Leftrightarrow x\oder y = y$
\end{itemize}
\end{definition}

\begin{definition}[Boolsche Algebra (s. 17)]
Notation $\mathfrak{A}=(A, \und , \oder , ', 0, 1)$
\begin{itemize}
  \item Trägermenge $A$
  \item Abbildungen: $\und, \oder : A\times A \rightarrow A$, assoziativ,
  kommutativ, distributiv
  \item Abbildung: $': A \rightarrow A$
  \item Neutrale/Inverse Elemente: $0, 1 \in A$
  \item \emph{Rechenregeln:}\begin{itemize}
    \item $x \oder x = x$, $x \und x = x$
    \item $x \oder (x \und y) = x$, $x \und (x \oder y) = x$
    \item $x \oder 0 = x$, $x \und 0 = 0$
    \item $x \oder 1 = 1$, $x \und 1 = x$
    \item $x \oder x' = 1$, $x \und x' = 0$
  \end{itemize}
  \item \emph{Es folgt:} \begin{itemize}
    \item $x \oder y = 1$ und $x \und y = 0$ $\Longrightarrow$ $y = x'$
    \item $(x \oder y)' = x' \und y'$ (de Morgan)
    \item $(x \und y)' = x' \oder y'$ (de Morgan)
  \end{itemize}
\end{itemize}
\end{definition}

\begin{definition}[Boolscher Term]
\begin{itemize}
  \item jedes $X_1, X_2,\ldots ,X_n$ ist ein Boolscher Term
  \item Wenn $P$, $Q$ Boolsche Terme sind, so auch $P'$, $(P\und Q)$, $(P
  \oder Q)$
\end{itemize}
\end{definition}

\subsection*{Äquivalenzrelationen}



\begin{definition}[Äquivalenzrelation (s. 7)]
Eine Äquivalenzrelation ist eine Relation, die 
\begin{itemize}
  \item reflexiv ($aRa$)
  \item symmetrisch ($aRb \Rightarrow bRa$)
  \item transitiv ($aRb \und bRc \Rightarrow aRc$)
\end{itemize}
ist.
\end{definition}

\begin{definition}[Äquivalenzklasse (s. 7)]
$(a)$ bezeichnet die Äquivalenzklasse zu a bezgl. der Äquivalenzrelation R und
ist definiert als $(a) = \{b \in A: aRb\}$
\end{definition}
\end{document}
